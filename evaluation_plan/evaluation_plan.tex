\documentclass[11pt]{article}

\usepackage{fullpage}
\usepackage{caption}
\usepackage{url}
\usepackage{gb4e}


\title{Evaluation Plan}
\author{Anthony Gentile \\ Lisa Gress}
\date{}


\begin{document}
\maketitle

\noindent {\bf Formalism of GREC semantic representation}\newline

The semantic annotations of the GREC corpus are centered around events related to gene expression and regulation. 
These events stem from verbs or nominalized verbs within the MEDLINE abstract and title. The first sentence is evaluated to see 
if it is on the topic gene regulation, if not, it goes on to the next sentence to do the same check. Of these sentences, 
the main verb of the sentence is located. This is the verb that describes the main event of the sentence. This event is annotated 
first, and then events from other verbs in the sentence thereafter.\newline

This means that events are not necessarily annotated in order. All events for all verbs in a sentence are annotated as 
long as the sentence is related to gene regulation. Related variables for these events are identified and then semantic roles are 
listed corresponding to these variables. It is possible for the identified semantic roles to correspond with another event as 
opposed to an identified trigger word. This means that there can be nested events. These annotations are available in an XML format as well as a standoff format.\newline

\noindent {\bf Different types of information encoded in the formalism}\newline

Within the GREC annotations, we are provided with event relationships from verbs. Within these we have two main types of information. 
First, we have concept classes used to label biological processes such as Repressor, Regulator, and Gene. Second, we have semantic role 
types such as Theme, Manner, Agent, and Location. These two types of information are tied to spans of the sentence by unique identifiers 
and structured within particular verb events that help us identify relationships.\newline

\noindent {\bf Atomic pieces of the representations}\newline

\noindent Using the standoff format of the annotations, the atomic pieces can be easily identified.\newline

\noindent For each MEDLINE abstract we have three files corresponding to the PMID of the abstract.

\begin{verbatim}
1885551.txt
1885551.a1
1885551.a2
\end{verbatim}

\noindent The .txt file contains the text of the MEDLINE abstract on two lines. The first being the abstract title, the second being the body of the abstract.\newline
\noindent The .a1 file contains named entity and event argument text spans. The .a2 file contains annotations relating to events.\newline

1885551.a1 (snippet)
\begin{verbatim}
T1	Repressor 0 23	Integration host factor
T2	SPAN 30 42	specifically
T3	Locus 46 60	multiple sites
T4	Transcription 115 128	transcription
T5	Regulator 181 202	a DNA-binding protein
T6	Gene 224 228	gene
\end{verbatim}

For each line we start with an ID for the annotation, followed by the annotation type, character offsets, and finally the surface text span of the annotation.\newline

1885551.a2 (snippet)
\begin{verbatim}
T26	GRE 24 29	binds
T27	GRE 106 114	inhibits
T28	GRE 208 220	participates
T29	Regulation 229 239	regulation
T30	GRE 527 532	binds
T31	Transcription 586 599	transcription
T35	GRE 825 832	binding
E1	GRE:T26 Agent:T1 Manner:T2 Destination:T3
E2	GRE:T27 Agent:T1 Theme:T4
E3	GRE:T28 Agent:T5 Location:T9,T11,T12 Descriptive-Agent:E4,T7,T8
E4	Regulation:T29 Theme:T6 Location:T9
\end{verbatim}

At the top of the .a2 files are annotations related to event trigger words. These are in the same format as .a1 files and often of the type GRE (Gene Regulation Event) which are often assigned to the top level verb trigger words.
After these event trigger word annotations, we have event annotations, for which the IDs start with 'E' instead of 'T'. Following this identifier is an event type and argument identifier pair followed by a list of semantic role 
and argument identifier pairs for the corresponding event.\newline

\noindent {\bf The atomic pieces to be used}\newline

We will use most of this information in our mapping process, however, we do not plan to consider the type information, such as GRE, Regulation, and other biological concept types in our work. 
We will be focused solely on the semantic role information provided. Additionally, we will need to do certain manipulations. One such manipulation will be to change offsets from being on the abstract level to a sentence level to mimic the MRS. \newline



\noindent {\bf Calculation of precision and recall}\newline

We hope to evaluate the MEDLINE abstract sentences independently using MRS and also with our gold standard GREC annotations.
By doing so we will be able to identify the sentences in which MRS and GREC annotations representations deviate and further delve into what those deviations are. 
We will measure the precision by taking the ouput of our mapping of the MRS format to GREC format and seeing where we have information that is not in the GREC annotations (gold standard). 
For recall, we will take the the output of our mapping to see which parts of the GREC annotations (gold standard) are not mapped to correctly.\newline

An example of such calculations are as follows:

\begin{verbatim}
GREC
"transcription" <9:22>
ARG1: Manner <0:8>
ARG2: Source <28:51> 

MRS
"_transcription_n_1_rel" <9:22>
ARG0: x3 <9:22> 
\end{verbatim}

From the MRS, the \verb|"_transcription_n_1_rel"| relation doesn't take any arguments so it doesn't provide us any relationships, whereas the GREC annotation (gold standard) provides for two. Given this sole context, we 
would have a precision of 0/2 and a recall of 0/2.\newline

\noindent {\bf Calculation of elements not mapped}\newline

In addition to quantifying instances, we also intend to list the deviations for inspection. We plan to count and list which parts of the MRS were not needed in the mapping procedure for futher analysis.
We hope to be able to provide a similar type of precision and recall calculation for this information as well, but are hestitant to show an example as we have yet to determine how we intend to structure that output. \newline

\noindent {\bf References}\newline

\noindent Paul Thompson, Syed A Iqbal, John McNaught and Sophia Ananiadou. 2009. Construction of an \\ \indent annotated corpus to support biomedical information extraction. {\em BMC Bioinformatics}, 10(1):349.



\end{document}



%As discussed in class, I think you'll likely have more traction in
%mapping the semantic roles than in trying to do anything with the
%ontological categories. For the semantic roles, you may well need to
%create a lexical resource which specifies the mapping of MRS role
%labels (ARGn) to GREC semantic roles.

%I'm also interested to see what kind of parsing accuracy/coverage
%you're going to get with the ERG, given the domain-specific vocab.

%-------------------------

%<ArticleTitle>
    %Positive and negative control of ompB transcription in Escherichia coli by cyclic AMP and the cyclic AMP receptor protein.
%</ArticleTitle>
%<Pagination>
  %<MedlinePgn>664-70</MedlinePgn>
%</Pagination>
%<Abstract>
  %<AbstractText>
      %The ompB operon encodes OmpR and EnvZ, two proteins that are necessary for the expression and osmoregulation of the OmpF and OmpC porins in Escherichia coli. 
      %We have used in vitro and in vivo experiments to show that cyclic AMP and the cyclic AMP receptor protein (CRP) directly regulate ompB. 
      %ompB expression in an ompB-lacZ chromosomal fusion strain was increased two- to fivefold when cells were grown in medium containing poor carbon sources or with added cyclic AMP. 
      %In vivo primer extension analysis indicated that this control is complex and involves both positive and negative effects by cyclic AMP-CRP on multiple ompB promoters. 
      %In vitro footprinting showed that cyclic AMP-CRP binds to a 34-bp site centered at -53 and at -75 in relation to the start sites of the major transcripts that are inhibited and activated, respectively, by this complex. 
      %Site-directed mutagenesis of the crp binding site provided evidence that this site is necessary for the in vivo regulation of ompB expression by cyclic AMP. 
      %Control of the ompB operon by cyclic AMP-CRP may account for the observed regulation of the formation of OmpF and OmpC by this complex (N. W. Scott and C. R. Harwood, FEMS Microbiol. Lett. 9:95-98, 1980).
  %</AbstractText>
%</Abstract>

%---------------

%<?xml version="1.0" encoding="utf-8"?>
%<!DOCTYPE Annotation PUBLIC "-//GREC//DTD GREC EVENT ANNOTATION//EN" "../GRECResources/GREC_event.dtd">
%<Annotation created="11/5/2009" creator="a6" annotates="http://www.ncbi.nlm.nih.gov/entrez/eutils/efetch.fcgi?db=pubmed&amp;retmode=xml&amp;rettype=medline&amp;id=1310090">
%<PubmedArticleSet>
%<PubmedArticle>
%<MedlineCitation>
%<PMID>1310090</PMID>
%<Article>
%<ArticleTitle>
%<sentence id="S1">Positive and negative control of <term sem="SPAN" id="T1" lex="ompB">ompB</term> transcription <term sem="SPAN" id="T2" lex="in_Escherichia_coli">in <term sem="Wild_Type_Bacteria" id="T3" lex="Escherichia_coli">Escherichia coli</term></term> by <term sem="Organic_Compounds" id="T4" lex="cyclic_AMP">cyclic AMP</term> and <term sem="Protein" id="T5" lex="the_cyclic_AMP_receptor_protein">the cyclic AMP receptor protein</term>.</sentence>
%<event id="E1">
%<type class="Transcription" />
%<Agent idref="T4" idref1="T5" />
%<Theme idref="T1" />
%<Location idref="T2" />
%<clue>Positive and negative control of ompB <clueType>transcription</clueType> in Escherichia coli by cyclic AMP and the cyclic AMP receptor protein.</clue>
%</event>
%</ArticleTitle>
%<Abstract>
%<AbstractText>
%<sentence id="S3"><term sem="Operon" id="T6" lex="The_ompB_operon">The ompB operon</term> encodes <term sem="Protein" id="T7" lex="OmpR_and_EnvZ">OmpR and EnvZ</term>, two proteins that are necessary for the expression and osmoregulation of <term sem="Protein" id="T8" lex="the_OmpF_and_OmpC_porins">the OmpF and OmpC porins</term> <term sem="SPAN" id="T9" lex="in_Escherichia_coli">in <term sem="Wild_Type_Bacteria" id="T10" lex="Escherichia_coli">Escherichia coli</term></term>.</sentence>
%<event id="E2">
%<type class="GRE" />
%<Agent idref="T6" />
%<Theme idref="T7" />
%<clue>The ompB operon <clueType>encodes</clueType> OmpR and EnvZ, two proteins that are necessary for the expression and osmoregulation of the OmpF and OmpC porins in Escherichia coli.</clue>
%</event>
%<event id="E3">
%<type class="Gene_Expression" />
%<Theme idref="T8" />
%<Location idref="T9" />
%<clue>The ompB operon encodes OmpR and EnvZ, two proteins that are necessary for the <clueType>expression</clueType> and osmoregulation of the OmpF and OmpC porins in Escherichia coli.</clue>
%</event>
%<event id="E4">
%<type class="Regulation" />
%<Theme idref="T8" />
%<Location idref="T9" />
%<clue>The ompB operon encodes OmpR and EnvZ, two proteins that are necessary for the expression and <clueType>osmoregulation</clueType> of the OmpF and OmpC porins in Escherichia coli.</clue>
%</event>
%<sentence id="S4">We have used in vitro and in vivo experiments to show that cyclic AMP and the cyclic AMP receptor protein (CRP) directly regulate ompB.</sentence>
%<sentence id="S5"><term sem="Gene" id="T11" lex="ompB">ompB</term> expression <term sem="SPAN" id="T12" lex="in_an_ompB-lacZ_chromosomal_fusion_strain">in <term sem=" Mutant_Bacteria" id="T13" lex="an_ompB-lacZ_chromosomal_fusion_strain">an ompB-lacZ chromosomal fusion strain</term></term> was increased <term sem="SPAN" id="T14" lex="two-_to_fivefold">two- to fivefold</term> <term sem="SPAN" id="T15" lex="when_cells_were_grown">when <term sem="Wild_Type_Bacteria" id="T16" lex="cells">cells</term> were grown</term> in <term sem="Organic_Compounds" id="T17" lex="medium_containing_poor_carbon_sources_or_with_added_cyclic_AMP">medium containing poor carbon sources or with added cyclic AMP</term>.</sentence>
%<event id="E5">
%<type class="Gene_Expression" />
%<Theme idref="T11" />
%<Location idref="T12" />
%<clue>ompB <clueType>expression</clueType> in an ompB-lacZ chromosomal fusion strain was increased two- to fivefold when cells were grown in medium containing poor carbon sources or with added cyclic AMP.</clue>
%</event>
%<event id="E6">
%<type class="GRE" />
%<Theme idref="E5" />
%<Condition idref="T17" />
%<Rate idref="T14" />
%<Temporal idref="T15" />
%<clue>ompB expression in an ompB-lacZ chromosomal fusion strain was <clueType>increased</clueType> two- to fivefold when cells were grown in medium containing poor carbon sources or with added cyclic AMP.</clue>
%</event>
%<sentence id="S6">In vivo primer extension analysis indicated that <term sem="SPAN" id="T18" lex="this_control">this control</term> is complex and involves both <term sem="SPAN" id="T19" lex="positive_and_negative">positive and negative</term> effects by <term sem="Protein_Complex" id="T20" lex="cyclic_AMP-CRP">cyclic AMP-CRP</term> <term sem="SPAN" id="T21" lex="on_multiple_ompB_promoters">on <term sem="Promoter" id="T22" lex="multiple_ompB_promoters">multiple ompB promoters</term></term>.</sentence>
%<event id="E7">
%<type class="GRE" />
%<Theme idref="T18" />
%<Descriptive-Theme idref="E8" />
%<clue>In vivo primer extension analysis indicated that this control is complex and <clueType>involves</clueType> both positive and negative effects by cyclic AMP-CRP on multiple ompB promoters.</clue>
%</event>
%<event id="E8">
%<type class="GRE" />
%<Agent idref="T20" />
%<Manner idref="T19" />
%<Location idref="T21" />
%<clue>In vivo primer extension analysis indicated that this control is complex and involves both positive and negative <clueType>effects</clueType> by cyclic AMP-CRP on multiple ompB promoters.</clue>
%</event>
%<sentence id="S7">In vitro footprinting showed that <term sem="Protein_Complex" id="T23" lex="cyclic_AMP-CRP">cyclic AMP-CRP</term> binds to <term sem="Locus" id="T24" lex="a_34-bp_site">a 34-bp site</term> centered at -53 and at -75 in relation to the start sites of <term sem="mRNA" id="T25" lex="the_major_transcripts">the major transcripts</term> that are inhibited and activated, respectively, by <term sem="Protein_Complex" id="T26" lex="this_complex">this complex</term>.</sentence>
%<event id="E9">
%<type class="GRE" />
%<Agent idref="T23" />
%<Destination idref="T24" />
%<clue>In vitro footprinting showed that cyclic AMP-CRP <clueType>binds</clueType> to a 34-bp site centered at -53 and at -75 in relation to the start sites of the major transcripts that are inhibited and activated, respectively, by this complex.</clue>
%</event>
%<event id="E10">
%<type class="Gene_Repression" />
%<Agent idref="T26" />
%<Theme idref="T25" />
%<clue>In vitro footprinting showed that cyclic AMP-CRP binds to a 34-bp site centered at -53 and at -75 in relation to the start sites of the major transcripts that are <clueType>inhibited</clueType> and activated, respectively, by this complex.</clue>
%</event>
%<event id="E11">
%<type class="Gene_Activation" />
%<Agent idref="T26" />
%<Theme idref="T25" />
%<clue>In vitro footprinting showed that cyclic AMP-CRP binds to a 34-bp site centered at -53 and at -75 in relation to the start sites of the major transcripts that are inhibited and <clueType>activated</clueType>, respectively, by this complex.</clue>
%</event>
%<sentence id="S8">Site-directed mutagenesis of the crp binding site provided evidence that this site is necessary for the <term sem="SPAN" id="T27" lex="in_vivo">in vivo</term> regulation of <term sem="Gene" id="T28" lex="ompB">ompB</term> expression by <term sem="Organic_Compounds" id="T29" lex="cyclic_AMP">cyclic AMP</term>.</sentence>
%<event id="E12">
%<type class="Regulation" />
%<Agent idref="T29" />
%<Theme idref="E13" />
%<Manner idref="T27" />
%<clue>Site-directed mutagenesis of the crp binding site provided evidence that this site is necessary for the in vivo <clueType>regulation</clueType> of ompB expression by cyclic AMP.</clue>
%</event>
%<event id="E13">
%<type class="Gene_Expression" />
%<Theme idref="T28" />
%<clue>Site-directed mutagenesis of the crp binding site provided evidence that this site is necessary for the in vivo regulation of ompB <clueType>expression</clueType> by cyclic AMP.</clue>
%</event>
%<sentence id="S9">Control of <term sem="Operon" id="T30" lex="the_ompB_operon">the ompB operon</term> by <term sem="Protein_Complex" id="T31" lex="cyclic_AMP-CRP">cyclic AMP-CRP</term> may account for the observed regulation of the formation of <term sem="Protein" id="T32" lex="OmpF_and_OmpC">OmpF and OmpC</term> by <term sem="Protein_Complex" id="T33" lex="this_complex">this complex</term> (N.W. Scott and C.R. Harwood, FEMS Microbiol. Lett. 9:95-98, 1980).</sentence>
%<event id="E14">
%<type class="GRE" />
%<Agent idref="T31" />
%<Theme idref="T30" />
%<clue><clueType>Control</clueType> of the ompB operon by cyclic AMP-CRP may account for the observed regulation of the formation of OmpF and OmpC by this complex (N.W. Scott and C.R. Harwood, FEMS Microbiol. Lett. 9:95-98, 1980).</clue>
%</event>
%<event id="E15">
%<type class="Regulation" />
%<Theme idref="E16" />
%<clue>Control of the ompB operon by cyclic AMP-CRP may account for the observed <clueType>regulation</clueType> of the formation of OmpF and OmpC by this complex (N.W. Scott and C.R. Harwood, FEMS Microbiol. Lett. 9:95-98, 1980).</clue>
%</event>
%<event id="E16">
%<type class="GRE" />
%<Agent idref="T33" />
%<Theme idref="T32" />
%<clue>Control of the ompB operon by cyclic AMP-CRP may account for the observed regulation of the <clueType>formation</clueType> of OmpF and OmpC by this complex (N.W. Scott and C.R. Harwood, FEMS Microbiol. Lett. 9:95-98, 1980).</clue>
%</event>
%</AbstractText>
%</Abstract>
%</Article>
%</MedlineCitation>
%</PubmedArticle>
%</PubmedArticleSet>
%</Annotation>

%-------------

%SENT: The ompB operon encodes OmpR and EnvZ, two proteins that are necessary for the expression and osmoregulation of the OmpF and OmpC porins in Escherichia coli.
%[ LTOP: h0
%INDEX: e2 [ e SF: prop ]
%RELS: < [ _the_q_rel<0:3> LBL: h4 ARG0: x3 [ x PERS: 3 NUM: sg ] RSTR: h5 BODY: h6 ]
 %[ compound_rel<4:15> LBL: h7 ARG0: e8 [ e SF: prop TENSE: untensed MOOD: indicative PROG: - PERF: - ] ARG1: x3 ARG2: x9 [ x IND: + ] ]
 %[ proper_q_rel<4:8> LBL: h10 ARG0: x9 RSTR: h11 BODY: h12 ]
 %[ named_rel<4:8> LBL: h13 CARG: "ompB" ARG0: x9 ]
 %[ "_operon/NN_u_unknown_rel"<9:15> LBL: h7 ARG0: x3 ]
 %[ "_encode_v_1_rel"<16:23> LBL: h15 ARG0: e16 [ e SF: prop TENSE: pres MOOD: indicative PROG: - PERF: - ] ARG1: x3 ARG2: x17 [ x PERS: 3 NUM: pl ] ]
 %[ udef_q_rel<24:38> LBL: h18 ARG0: x17 RSTR: h19 BODY: h20 ]
 %[ proper_q_rel<24:28> LBL: h21 ARG0: x22 [ x PERS: 3 NUM: sg IND: + ] RSTR: h23 BODY: h24 ]
 %[ named_rel<24:28> LBL: h25 CARG: "OmpR" ARG0: x22 ]
 %[ _and_c_rel<29:32> LBL: h27 ARG0: x17 L-INDEX: x22 R-INDEX: x28 [ x PERS: 3 NUM: sg IND: + ] ]
 %[ proper_q_rel<33:38> LBL: h29 ARG0: x28 RSTR: h30 BODY: h31 ]
 %[ named_rel<33:38> LBL: h32 CARG: "EnvZ" ARG0: x28 ]
 %[ subord_rel<39:157> LBL: h1 ARG0: e34 [ e SF: prop TENSE: untensed MOOD: indicative PROG: - PERF: - ] ARG1: h35 ARG2: h36 ]
 %[ udef_q_rel<39:42> LBL: h37 ARG0: x38 [ x PERS: 3 NUM: pl ] RSTR: h39 BODY: h40 ]
 %[ card_rel<39:42> LBL: h41 CARG: "2" ARG0: e43 [ e SF: prop TENSE: untensed MOOD: indicative ] ARG1: x38 ]
 %[ "_protein_n_1_rel"<43:51> LBL: h41 ARG0: x38 ]
 %[ "_necessary_a_for_rel"<61:70> LBL: h41 ARG0: e44 [ e SF: prop TENSE: pres MOOD: indicative PROG: - PERF: - ] ARG1: x38 ARG2: x45 [ x PERS: 3 ] ]
 %[ _the_q_rel<75:78> LBL: h46 ARG0: x45 RSTR: h47 BODY: h48 ]
 %[ udef_q_rel<79:89> LBL: h49 ARG0: x50 [ x PERS: 3 NUM: sg ] RSTR: h51 BODY: h52 ]
 %[ "_expression_n_1_rel"<79:89> LBL: h53 ARG0: x50 ]
 %[ udef_q_rel<90:108> LBL: h54 ARG0: x55 [ x PERS: 3 NUM: sg ] RSTR: h56 BODY: h57 ]
 %[ _and_c_rel<90:93> LBL: h58 ARG0: x45 L-INDEX: x50 R-INDEX: x55 ]
 %[ "_osmoregulation/NN_u_unknown_rel"<94:108> LBL: h59 ARG0: x55 ]
 %[ _of_p_rel<109:111> LBL: h58 ARG0: e60 [ e SF: prop ] ARG1: x45 ARG2: x61 [ x PERS: 3 NUM: pl ] ]
 %[ _the_q_rel<112:115> LBL: h62 ARG0: x61 RSTR: h63 BODY: h64 ]
 %[ compound_rel<116:136> LBL: h65 ARG0: e66 [ e SF: prop TENSE: untensed MOOD: indicative PROG: - PERF: - ] ARG1: x61 ARG2: x67 [ x PERS: 3 NUM: pl ] ]
 %[ udef_q_rel<116:129> LBL: h68 ARG0: x67 RSTR: h69 BODY: h70 ]
 %[ proper_q_rel<116:120> LBL: h71 ARG0: x72 [ x IND: + ] RSTR: h73 BODY: h74 ]
 %[ named_rel<116:120> LBL: h75 CARG: "OmpF" ARG0: x72 ]
 %[ _and_c_rel<121:124> LBL: h77 ARG0: x67 L-INDEX: x72 R-INDEX: x78 [ x IND: + ] ]
 %[ proper_q_rel<125:129> LBL: h79 ARG0: x78 RSTR: h80 BODY: h81 ]
 %[ named_rel<125:129> LBL: h82 CARG: "OmpC" ARG0: x78 ]
 %[ "_porins/NNS_u_unknown_rel"<130:136> LBL: h65 ARG0: x61 ]
 %[ _in_p_rel<137:139> LBL: h84 ARG0: e85 [ e SF: prop TENSE: untensed MOOD: indicative ] ARG1: x38 ARG2: x86 [ x PERS: 3 NUM: sg ] ]
 %[ udef_q_rel<140:157> LBL: h87 ARG0: x86 RSTR: h88 BODY: h89 ]
 %[ compound_rel<140:157> LBL: h90 ARG0: e91 [ e SF: prop TENSE: untensed MOOD: indicative PROG: - PERF: - ] ARG1: x86 ARG2: x92 [ x IND: + ] ]
 %[ proper_q_rel<140:151> LBL: h93 ARG0: x92 RSTR: h94 BODY: h95 ]
 %[ named_rel<140:151> LBL: h96 CARG: "Escherichia" ARG0: x92 ]
 %[ "_coli/NN_u_unknown_rel"<152:157> LBL: h90 ARG0: x86 ] >
%HCONS: < h0 qeq h1 h5 qeq h7 h11 qeq h13 h19 qeq h27 h23 qeq h25 h30 qeq h32 h35 qeq h15 h36 qeq h84 h39 qeq h41 h47 qeq h58 h51 qeq h53 h56 qeq h59 h63 qeq h65 h69 qeq h77 h73 qeq h75 h80 qeq h82 h88 qeq h90 h94 qeq h96 > ]
