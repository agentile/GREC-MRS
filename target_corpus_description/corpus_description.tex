\documentclass[11pt]{article}

\usepackage{fullpage}

\title{Target Corpus Description}
\author{Anthony Gentile \\ Lisa Gress}
\date{}


\begin{document}
\maketitle

% The Gene Regulation Event Corpus (GREC) is a compilation of 240 MEDLINE (http://www.nlm.nih.gov/bsd/pmresources.html) abstracts. 
% 167 of these abstracts are on the subject of E. coli species, while the remaining 73 are on the subject of the Human species.
% The purpose of this corpus is for training of information extraction particularly those attempting to extract events from the 
% biomedical domain. 

% The corpus is annotated by hand by biologists. These annotations focus on verb centered events, while variables of these events are labeled
% from a fixed set of thirteen semantic roles.

% 13 Semantic roles (really 12, but descriptive is broken down into two sub roles)
% AGENT, THEME, MANNER, INSTRUMENT, LOCATION, SOURCE, DESTINATION, TEMPORAL, CONDITION, RATE, DESCRIPTIVE-AGENT, DESCRIPTIVE-THEME, PURPOSE

% Average inter-annotator agreement rates fall within the range of $66%$ - $90%$. 

% http://www.biomedcentral.com/1471-2105/10/349
% Paul Thompson, Syed A Iqbal, John McNaught and Sophia Ananiadou
% Construction of an annotated corpus to support biomedical information extraction
% BMC Bioinformatics 2009, 10:349 doi:10.1186/1471-2105-10-349

% http://www.nactem.ac.uk/GREC/
% http://www.nactem.ac.uk/GREC/xml.php
% http://www.nactem.ac.uk/GREC/Event_annotation_guidelines.pdf

% What is the source of the texts to be annotated?

% How much annotated text is available?

% What information do the annotations capture?

% How was the annotation scheme designed?

% What was the original domain in which the annotation scheme was designed?

% What was the original purpose of the annotations? (e.g., some particular task)

% How were the annotations created (by hand, via a grammar, other)?

% Was inter-annotator agreement measured? What was the result?

% How closely constrained are the annotations by linguistic (syntactic, morphological) structure?

% How can the annotation scheme, annotated texts and source texts be cited?

\end{document}
